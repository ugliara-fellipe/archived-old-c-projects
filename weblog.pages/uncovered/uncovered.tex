%
% Copyright 2021 Fellipe Augusto Ugliara
%
% Content on this file is licensed under the Creative Commons Attribution 
% 4.0 International License. To view a copy of this license: visit 
% http://creativecommons.org/licenses/by/4.0/, or search for the CONTENT 
% file at https://github.com/ugliara-fellipe/weblog.pages
%

\documentclass[11pt]{article}
\usepackage[a4paper, left=0.5in, right=0.5in, top=0.95in, bottom=0.90in]{geometry} %showframe

\usepackage[T1]{fontenc}
\usepackage[utf8]{inputenc}

\usepackage{hyphenat}

\usepackage{dirtytalk}
\usepackage{csquotes}
\usepackage{lipsum}
\usepackage{fixltx2e}
\usepackage{tocloft}
\usepackage{geometry}
\usepackage{anyfontsize}
\usepackage{multirow}
\usepackage{tabularx}
\usepackage{fix-cm}

\setlength{\cftsecindent}{0pt}% Remove indent for \section
\setlength{\cftsubsecindent}{0pt}% Remove indent for \subsection
\setlength{\cftsubsubsecindent}{0pt}% Remove indent for \subsubsection

\usepackage{blindtext}
\usepackage{multicol}
\usepackage{color}
\setlength{\columnseprule}{0.1pt}
\def\columnseprulecolor{\color{black}}
\setlength{\columnsep}{1cm}

\renewcommand{\thefootnote}{}

\usepackage{lastpage}
\usepackage{fancyhdr}
\pagestyle{fancy}
\fancyhf{}
\fancyhead[L]{\small{Uncovered. Tabloid Ruleset.} \normalsize}
\fancyhead[R]{\small \today \normalsize}
\fancyfoot[L]{\small Copyright 2021 Fellipe Augusto Ugliara. Licensed under the CC BY 4.0 \normalsize}
\fancyfoot[R]{\small \thepage\ of \pageref{LastPage} \normalsize}
\renewcommand{\footrulewidth}{0.1pt}
\renewcommand{\headrulewidth}{0.1pt}
\setlength{\headsep}{0.5cm}
\fancypagestyle{plain}{
	\renewcommand{\headrulewidth}{0.1pt}
	\renewcommand{\footrulewidth}{0.1pt}
	\setlength{\headsep}{0.55cm}
	\fancyhf{}
	\fancyhead[L]{Storytelling Game by Fellipe Augusto Ugliara}
	\fancyhead[R]{Vol. 1. No. 1. Pages. \pageref{LastPage}. \$. Free.}
	\fancyfoot[L]{\small Copyright 2021 Fellipe Augusto Ugliara. Licensed under the CC BY 4.0 \normalsize}
	\fancyfoot[R]{\small \thepage\ of \pageref{LastPage} \normalsize}}

\title{\huge{Uncovered}}
\author{Fellipe Augusto Ugliara}
\date{\today}

\makeatletter         
\renewcommand\maketitle{
	{\raggedright
		\begin{tabularx}{\textwidth}{@{}l@{\extracolsep{\fill}}r@{}}
			\multirow{3}{*}{\fontsize{60pt}{60pt}\selectfont Uncovered} & 
			\vspace{-0.4cm}  \\ 
			& {\huge Tabloid Ruleset}  \\ [0.25cm]
			& {\huge \@date}
		\end{tabularx}
	}
}
\makeatother

%-----------------------------------------------------

\begin{document}

\maketitle

\noindent\rule{\textwidth}{0.1pt}
\begin{multicols}{3}
	\small	\tableofcontents \normalsize
\end{multicols}
\noindent\rule{\textwidth}{0.1pt}

\begin{multicols}{2}
\setlength{\columnseprule}{0pt}

\part{Introduction}

\lipsum[1]

\section{Storytelling Game}

A narration is a story told by someone about something. It is a succession of facts, associated with places and time intervals. This sequence of events is called a plot. The one who tells the story is the narrator, he reports the situations experienced by the characters in the narrative.

Stories are generally transmitted by mixed, verbal and non-verbal communication. In verbal communication, whether oral or written, the main elements used are words. In the written format, communication takes place through books, letters, newspapers. While in oral format, it takes place through conversations, speeches, the radio.

Non-verbal communications such as body language, tone of voice, use of images, use of gestures enhance the way in which the verbal transmission of narrations takes place.

An exception is the use of sign language together with other non-verbal elements, which allows transmitting and enhance narrations without using verbal communication. And in this case, the transmission of the narration cannot be considered a mixed communication.

Communicating a narration usually places the audience in a passive situation. The narrator is solely responsible for its development, and the interaction with the public is a one-way street, starting with the narrator and ending with the audience that receiving the narration.

Narrations can be freely invented or adapted by those who communicate them, without any intervention from the audience to which they are broadcast. And various verbal and non-verbal elements can be used to convey the narration as the narrator sees fit.

Games are activities that have rules and players. The player is the one who participates in the game, and the rules are the definition of how the player interacts with other players and with the game's accessories. Miniatures, board, cards, and other things used during games are game accessories.

Unlike narrations, in which the narrator can freely invent and adapt. In games, participants are limited by a set of rules. On the other hand, in the games, everyone involved acts actively during the matches. While in narrations only the narrator is actively involved with the story, the audience passively acts with the narrative.

Storytelling games are a mixture of game features with narration. They are an iterative narration, in which each participant of the audience takes control of some of their characters. The narrator conducing the plot and share the responsibility to control of some characters with the players.

In storytelling games, the audience's passive role is replaced by an active player's participation. In which, players receive a share of the narrator's freedom, deciding their characters' actions. This performance of players is guided by a set of rules, which define how they interact with the narration.

The recommended number of participants for a storytelling game is two to five players plus one narrator. In games with fewer than four players, allowing some to use more than one character can be a good enough adopted strategy.

And in games with more than five players, an alternative is to form different groups each with its own narrator, and end the games by unifying the narrations, in a narration with all the players and a group of narrators.

\subsection{The Hum}

%histori de alguem que escuta sons vindo do ceu em seu apartamento

\lipsum[1]


\section{Free Content}

There are a lot of paid storytelling games out there, and they don't come cheap. This material, however, is free. Both Uncovered and the Tabloid Ruleset were designed to be affordable, and charging for them would be extremely contradictory.

This work is licensed under the Creative Commons Attribution 4.0 International License (CC BY 4.0). To view a copy of this license, visit http://creativecommons.org/licenses/by/4.0/ or send a letter to Creative Commons, PO Box 1866, Mountain View, CA 94042, USA.

In summary, the content of this work can be: shared, which consists in being able to copy and redistribute the material in any support or format; and adapted, which means being able to remix, transform, and create from the material for any purpose.

The only requirements are: give proper credit to the author of this work, provide a link to the CC BY 4.0 license, and indicate if changes have been made to the material. These requirements must be met in any reasonable circumstance, but in no way that suggests that the author supports anyone, or that someone's use of the material.

\part{Tabloid Ruleset}

Each game has its own set of rules, they define what participants can and cannot do during a game match. They even specify how matches start, and define what conditions are necessary for the game to end. If a rule is used in various contexts, it is considered a game mechanic, and can be devised independently of the games that use it.

Some notable mechanics are: rolling dice, to compare the sum of values to a given number; moving pieces around a board; buy cards from a deck; consume a certain conquered resource; combine a given move with another participant.

Game mechanics can be separated into groups, identifying the properties each one has. Therefore, the strengths and weaknesses of these mechanics can be compared. These comparisons help identify which mechanics best attribute certain characteristics to a set of rules.

The Tabloid Ruleset was designed with the intention to use non-probabilistic rules, keep the game dynamic, and only need paper and pen to play. Thus, dice, cards, miniatures, and other accessories used in traditional games were discarded.

Apparently these choices might seem limiting. On the other hand, they are a way of giving more attention to the narrative, valuing the players' strategies, and allowing the game to be accessible to those who do not have the aforementioned accessories.

The mechanics adopted in the Tabloid Ruleset are: characters evolution story based; limited use of certain actions; and the simultaneous resolution of the turn actions, both those chosen by the narrator and those deliberated by the players.

This set of rules is aimed at games that mix a bit of fiction with realistic narration. Stories involving: investigation of crimes, supernatural mysteries, secret societies, mystical creatures, conspiracies, archaeological explorations are appropriately represented in the Tabloid Ruleset.

However, narrations that are: loaded with magic, superpowers, cinematic combats, high-tech, or those that are deeply realistic, are not properly represented in this rule set. Imagining a line that starts with realistic narrations and ends with fictional ones. The Tabloid Ruleset is best used to represent narrations that are far from the beginning of the line, but that do not exceed its midpoint.

\section{Definitions}

\lipsum[1]

\subsection{Passive}

\lipsum[1]

\subsection{Active}

\lipsum[1]

\section{Perform Actions}

\lipsum[1]

\subsection{Free}

\lipsum[1]

\subsection{Disputes}

\lipsum[1]

\subsection{Agreements}

\lipsum[1]

\subsection{Recovery}

\lipsum[1]

\section{Ancestral Ruins}

\lipsum[1]

\subsection{Players}

\lipsum[1]

\subsection{Storyteller}

\lipsum[1]

\subsection{Let's Jam}

\lipsum[1]

\part{Characters}

\lipsum[1]

\section{Misfortunes}

\lipsum[1]

\subsection{Injuries}

\lipsum[1]

\subsection{Fatigue}

\lipsum[1]

\subsection{Madness}

\lipsum[1]

\subsection{Confusion}

\lipsum[1]

\section{Traits}

\lipsum[1]

\subsection{Positives}

\lipsum[1]

\subsection{Negatives}

\lipsum[1-2]

\section{Skills}

\lipsum[1]

\section{Equipments}

\lipsum[1]

\section{Evolution}

\lipsum[1]

\section{Archetypes}

\lipsum[1]

\subsection{News Reporter}

\lipsum[1]

\subsection{Police Investigator}

\lipsum[1]

\subsection{The Professor}

\lipsum[1]

\part{Capital City}

\lipsum[1]

\section{Places}

\lipsum[1]

\section{Myths}

\lipsum[1]

\section{Conspiracies}

\lipsum[1]

\part{Suggestions}

\lipsum[1]

\section{Tabloid Journalism}

\lipsum[1]

\section{Pulp Magazines}

\lipsum[1]

\end{multicols}

\end{document}