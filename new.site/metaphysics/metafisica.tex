%
% Copyright 2021 Fellipe Augusto Ugliara
%
% Content on this file is licensed under the Creative Commons Attribution 
% 4.0 International License. To view a copy of this license: visit 
% http://creativecommons.org/licenses/by/4.0/, or search for the CONTENT 
% file at https://github.com/ugliara-fellipe/weblog.pages
%

\documentclass[11pt]{article}
\usepackage[a4paper, left=1.11in, right=1.11in, top=1.60in, bottom=1.55in]{geometry}

\usepackage{titling}
\setlength{\droptitle}{-8.5em}

\usepackage[T1]{fontenc}
\usepackage[utf8]{inputenc}

\usepackage[portuguese]{babel}
\usepackage{hyphenat}

\usepackage{dirtytalk}
\usepackage{csquotes}

\renewcommand{\thefootnote}{}

\usepackage{lastpage}
\usepackage{fancyhdr}
\pagestyle{fancy}
\fancyhf{}
\fancyhead[L]{\small Identificações Aleatórias da Realidade \normalsize}
\fancyhead[R]{\small 20 de Junho de 2021 \normalsize}
\fancyfoot[L]{\small Copyright 2021 Fellipe Augusto Ugliara. Licenciado sob a CC BY 4.0 \normalsize}
\fancyfoot[R]{\small \thepage\ de \pageref{LastPage} \normalsize}
\renewcommand{\footrulewidth}{0.5pt}
\fancypagestyle{plain}{
	\renewcommand{\headrulewidth}{0pt}
	\fancyhf{}
	\fancyfoot[L]{\small Copyright 2021 Fellipe Augusto Ugliara. Licenciado sob a CC BY 4.0 \normalsize}
	\fancyfoot[R]{\small \thepage\ de \pageref{LastPage} \normalsize}}

\title{Identificações Aleatórias da Realidade}
\author{Fellipe Augusto Ugliara}
\date{20 de Junho de 2021}

\begin{document}
	\maketitle
	
	\section{Introdução} \label{pt-s1}
	
	Um padre fala aos fiéis a respeito da criação na missa da manhã. Acadêmicos discutem a composição da matéria. O médium comunica-se com os espíritos em reuniões banhadas a meia luz. Os monges em posição de lótus meditam buscando o Nirvana. Um idoso sentado na praça questiona se os pássaros sabem que horas são.
	
	Os cenários citados são carregados de símbolos, teorias, histórias, percepções, observações, suposições acerca de perguntas fundamentais como: \say{O que é o ser?}, \say{O que é o mundo?}, \say{Qual a origem de tudo?}, \say{O que são os pensamentos?}, \say{O que é o tempo?}.
	
	Essas perguntas estão presentes em contextos variados; e as respostas, atribuídas a elas, constantemente acabam em algum impasse. Investigando essas perguntas e respostas, uma característica recorrente é observada: ao atribuir uma resposta a alguma dessas perguntas, novas questões podem ser produzidas.
	
	Qual a origem de tudo? Se Deus é uma boa resposta, então \say{Qual a origem de Deus?} é uma nova boa pergunta. O que é o ser? Se ele for um grupo de átomos, então \say{O que é o átomo?} será uma nova questão a se considerar. Sempre é possível questionar o que veio antes, e o que acontecerá depois.
	
	Eliminar essa característica recorrente requer uma nova compreensão do que seria a realidade. É preciso romper com ideias bem definidas, e remover a solidez aparente do mundo, para ser possível vislumbrar um eterno mar de aleatoriedade. 
	
	\section{Potencialidade} \label{pt-s2}
	
	É possível identificar o vermelho, o amarelo, ou o azul, e cada uma dessas opções é a identificação de algo que também é identificado como cor. É factível identificar que algo pode ou não existir, e essas duas opções são também identificadas como o conceito de existência. Identificar raposas, ursos, coelhos, cavalos, gatos é possível e cada uma dessas opções é também identificada como animal, criatura.
		
	\textit{Potencialidade} é a possibilidade de algo, característica, objeto, criatura, ideia, conceito, pensamento, lembrança, sentimento, acontecimento, não importa o que possa ser, seja identificado. Na próxima seção é definida a ideia do que seria uma \textit{identificação}, e como o ser é interpretado com base nessa ideia.
	
	\section{Identificação} \label{pt-s3}
	
	A elaboração de uma resposta, a respeito do que seria o ser, inicia com a seguinte ideia: os seres, sejam o que forem, tem a potencialidade de serem identificados. Uma \textit{identificação} é a distinção de um ser em relação a outros seres. Para representar todo o potencial de seres identificáveis, será empregado o termo \textit{realidade}, e para representar uma ocorrência das identificações de todos esses potenciais seres identificáveis, será utilizado o termo \textit{interpretação}.
	
	Para entender a essência das identificações é razoável começar analisando a quantidade de identificações que podem ocorrem em uma interpretação. A ocorrência das seguintes quantidades precisam ser analisadas: duas identificações ou mais; uma identificação; nenhuma identificação.
	
	Nas interpretações com duas identificações ou mais, se um ser é identificado em relação aos demais seres, esses também serão identificados relativamente ao ser. Não importando qual seja o ser alvo identificado, a distinção sempre será realizável em ambas as direções. Será possível distinguir: o ser, dos outros seres; e os outros seres, do ser.
	
	Nas interpretações em que somente um ser é identificado, surge a perguntar: O ser foi identificado em relação a quê? É possível supor: que exista um meio onde o ser reside e pode ser identificado, ou que existem seres não identificados utilizados como referência para identificar o ser.
	
	No primeiro caso, o meio de residência pode ser entendido como outro ser. É possível distinguir: o ser, do meio onde reside; e o meio de residência, do ser. Portanto, o meio onde o ser reside, torna-se um ser identificado. Esse caso corresponde então a identificação de dois seres, e não de um único ser.
	
	No segundo caso, o ser é identificado em relação ao restante dos seres não identificados, que passam a ser identificados em relação ao ser. O restante dos seres não identificados se tornam um ser identificado. Assim sendo esse caso também corresponde a identificação de dois seres, e não de um único ser.
	
	Nas interpretações que não existem identificações, ainda está presente toda a potencialidade de seres identificáveis. Nesse caso a interpretação corresponde diretamente com a definição de realidade, e não pode ser considerada uma interpretação, visto não ter ocorrido nenhuma identificação.
	
	Portanto, em qualquer interpretação ocorre no mínimo duas identificações. Esse fato permite que, uma definição envolvendo comparação de diferentes identificações, possa ser usada no contexto de qualquer interpretação.
	
	Assim sendo, o ser é definido como uma identificação parcial da realidade, distinguível das demais identificações que compõem uma mesma interpretação. Essa resposta ainda possui lacunas abertas.
	
	Na seção \ref{pt-s4} será discutido se a realidade pode ser separada em potenciais seres identificáveis. Já na seção \ref{pt-s5} será visto como as identificações percebem a interpretação onde ocorrem.
	
	\section{Composição} \label{pt-s4}
	
	Para definir o que é uma identificação parcial da realidade, será explorado o conceito de como os seres identificados representam composições. Uma interpretação do que são seres unidos e separados.
	
	Supondo que um ser é formado por dois outros seres. Se eles são separados o ser formado deixaria de existir? Ou o inverso, se dois seres unidos formam outro ser, esse outro ser existe antes da união dos seres separados?

	Identificar um ser em outro não quer dizer que exista alguma separação entre eles. Isso quer dizer que os seres de uma interpretação podem representar identificações maiores ou menores da realidade.
	
	A realidade não muda, mas dependendo de como a composição das identificações ocorre, interpretações diferentes são produzidas. Uma interpretação é um modo de representar a realidade através de uma determinada composição de identificações.
	
	O que não ocorre são interpretações que possuem parciais da realidade não identificadas. Ao identificar um ser, o restante da realidade também se torna identificada, como foi explicado na seção anterior.
	
	Somente a ideia de composição não é suficiente para esclarecer como as identificações percebem a interpretação onde ocorrem. Para isso, é necessário definir também, como uma identificação pode ser representada empregando as próprias identificações.

	\section{Consistência} \label{pt-s5}
	
	A consistência é uma característica associada às interpretações. Dependendo de como as identificações ocorrem, a interpretação será \textit{consistente} ou \textit{inconsistente}. As interpretações consistentes serão aquelas percebidas por alguma ou algumas de suas identificações.
	
	Uma identificação ocorre de modo único em uma interpretação. Se ela puder ocorrer de duas formas ou mais, em uma mesma interpretação, ela seria um potencial identificável, e não uma identificação.
	
	Não é necessário precisar quais identificações precisam ocorrer para que a interpretação seja consistente. Todas as possíveis interpretações consistentes e inconsistentes ocorrem simultaneamente. Cada interpretação dessas, representa uma das possíveis composições de identificações que podem ocorrer.
	
	O que caracteriza a consistência da interpretação é o fato de alguma ou algumas de suas identificações conseguirem perceber essa interpretação. Nem todas as interpretações serão consistentes, porém isso não significa que elas não existam. Essas interpretações inconsistentes existem, mas não são percebidas por suas identificações. 
	
	Uma identificação, é uma parcial da realidade. Pode-se afirmar que essa identificação \textit{percebe} a interpretação na qual ocorre: se uma parte menor dessa parcial for uma composição de identificações que represente, limitadamente, a interpretação na qual a identificação ocorreu.
	
	Limitar as \textit{percepções} segue a mesma ideia de restringir as interpretações. Cada interpretação é uma forma restrita da realidade, e cada percepção é uma forma limitada da interpretação. Caso a interpretação não fosse uma forma restrita, ela seria a própria realidade, e interpretações não seriam possíveis. O mesmo ocorre com as percepções. Se elas não fossem uma forma limitada da interpretação, elas seriam a própria interpretação, e percepções não seriam viáveis.
	
	As percepções também são formadas por uma composição de identificações e essa composição faz parte da composição de identificações, que forma a interpretação. As identificações das percepções não estão separadas ou diferem das identificações das interpretações, empregar essa separação é conveniente para definir o que conhecemos como pensamentos, sentimentos, ideias. Não podemos separar efetivamente as percepções da realidade, elas também são composições de identificações em interpretações da realidade, mas é conveniente denominar essas composições de identificações como percepções.
	
	É importante destacar que uma interpretação inconsistente pode ser vista como consistente se somente uma parte de suas identificações for considerada. Isso explica como é possível elaborar conceitos sobre interpretações diferentes da percebida pelas identificações, e explica também como é possível elaborar ideias sobre a potencialidade, mesmo sem poder percebê-la completamente.

	\section{Ordenação} \label{pt-s6}
	
	Ordenar é definir que as identificações podem ser organizadas uma após a outra, é aceitar que elas possuem alguma relação de precedência. Ordenar é um conceito que não pode ser aplicado à realidade, dado que tudo é potencial e não existem seres identificados para serem ordenados.
	
	Poder ordenar é importante nas interpretações, pois se não for possível perceber a ordenação, não seria factível definir conceitos sequenciais como o tempo ou a contagem. Portanto, para que algumas interpretações sejam consistentes, existem dentre suas identificações as que são percebidas em alguma categoria de ordem.
	
	A ordenação é similar à composição. Um número pode ser dividido em partes menores, e percebemos cada parte como uma identificação. Ele também pode ser unido a outro número para formar uma parte maior que também podemos perceber através de sua identificação. 
	
	O tempo é similar, ele pode ser percebido como a identificação do minuto, mas ele também pode ser separado em duas partes de trinta segundos que são identificações igualmente percebidas. Essa identificação das duas partes de trinta segundos não separa o minuto realmente, ele só é percebido de outro modo.
	
	Perceber identificações em alguma relação de ordem não significa que elas ocorrerem uma após a outra, pois todas as identificações ocorrem simultaneamente na interpretação. 
	
	Nas interpretações em que essas identificações ocorrem de modo a sugerir uma ordenação, elas poderão tornar as interpretações consistentes. Nas que ocorrem de modo a não sugerir uma ordenação, elas poderão levar a interpretações inconsistentes.

	\section{Conclusão} \label{pt-s7}
	
	As interpretações ocorrem em \textit{pulsações} da realidade, o \textit{pulsar} não é um evento sequencial ou temporal, mas ele acontece sem cessar perpetuamente. A realidade pulsa continuamente, e as interpretações ocorrem simultaneamente a cada pulso. Podendo assim, com base nas ocorrências aleatórias de identificações, produzir interpretações consistentes.
	
	Compreender desse modo: o que é o ser, o que são os pensamentos, o que é a existência, derruba a ideia de que existe um começo, e um final. Identificar as partes da potencialidade: é limitar as possibilidades; é delinear uma interpretação; é supor um começo e fim para cada uma dessas partes. É pura ilusão acreditar que algum conjunto de identificações seja a própria realidade. Uma interpretação, um desses conjuntos de identificações, é somente uma representação limitada das possíveis representações da potencialidade.
	
	\footnote{Este trabalho está licenciado sob a Licença Atribuição 4.0 Internacional Creative Commons. Para visualizar uma cópia desta licença, visite http://creativecommons.org/licenses/by/4.0/ ou mande uma carta para Creative Commons, PO Box 1866, Mountain View, CA 94042, USA.}
\end{document}
